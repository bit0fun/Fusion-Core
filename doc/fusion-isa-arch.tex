\documentclass[letterpaper, 11pt]{article}
\usepackage[T1]{fontenc}
\usepackage{inconsolata}
\usepackage{fontspec}
\defaultfontfeatures{Mapping=tex-text,Scale=MatchLowercase}
\setmainfont{Hack}
\usepackage{longtable}
\usepackage{graphicx}
\usepackage[left=1in, right=1in, top=1.00in, bottom=1.00in]{geometry}
\usepackage{subfigure}
\usepackage{hyperref}
\usepackage{amssymb}
\hypersetup{colorlinks=false,linktoc=all}


\title{Fusion-Core ISA Definition: Revision 0.2}
\author{Dylan Wadler}

\begin{document}
\maketitle
\newpage
\tableofcontents


\newpage
\part{Administrative}
\section{Change log}
\paragraph{Version 0.2} Added comprehensive list of Instruction usage for programmers. Changed opcode definitions to create simpler
decode, with availability to create new instructions from the existing opcodes. Including more information about special purpose registers.
\paragraph{Version 0.1} Initial Definition of the Instruction Set Architecture

\section{Introduction}

\subsection{About}
\paragraph{Introduction}The Fusion-Core ISA is dedicated to creating an easily expandable architecture without having to 
recompile a program's binary. By use of defining an easy interface with a simple core instruction set, this allows for
more freedom in implementation. High end processors and microcontrollers would only require slight variations in
configuration, as their core would remain identical save for easy to maintain and scalable co-processors.
\paragraph{Main Ideas}The architecture is Big endian, with a core instruction set that is RISC, but the co-processors do not need to adhere to the RISC philosophy. This allows for more flexibility in design, and possibly faster core clock speeds as the pipeline would depend on smaller amounts of logic. Only the instructions provided in this document are to be implemented in the main processor. The co-processors defined in this document are recommended, but not required for normal function. Co-processor documentation is to be provided by the creator, and should adhere to the standards of clarity and conciseness such that it can be easily implemented from the documentation alone in a HDL. 
\paragraph{64 Bit instructions:}At this moment in time, the Fusion-Core ISA is only a 32 bit ISA. Due to the focus on co-processors,
older implementations could easily be modified to include 64 bit operations.
\paragraph{Co-Processors}The Co-Processor interface is currently defined by setting the MSB within the OP Code field of 
an instruction, to decrease complexity of the Decode unit. In doing so, this allows for co-processor code to be written in the same memory space as the main processor code. In the current iteration, up to 32 different co-processors can be used, with the option for dynamic or static allocation of the OP Codes. The interface for co-processors is explained further in the dedicated section.

\subsection{Goals}
\paragraph{}The main goal is to provide an architecture with a simple decoding unit and the ability to utilize a single binary for all implementations of the architecture.
In order to do this, a strict implementation of the main core is outlined in this document, along with registering co-processors to provide consistency. 
Mechanisms to allow for co-processor instructions to be executed in software are also available; more information is available in the respected section.
\subsection{Conventions}
\paragraph{Document Conventions:}Example code will be shown with \texttt{monospace} text. General purpose registers will be denoted with
\$R\# where \# is the number of the register. Special purpose registers will be written with \textbf{bold} text.
%\paragraph{Naming Conventions:}The name of input signals will have "\_in" after the signal name, with "\_out" after output signals.
%This is mainly used in the verilog example implementation. If a naming convention is not globally used, it will be stated in the individual section that it pertains to. 
\begin{center}
	\line(1,0){250}
\end{center}
\clearpage
\part{Programming Information}
\section{Register File Definitions}
\paragraph{}This section goes over the different registers available in the ISA. Each register file name begins with 
"REGF", such as the first General Purpose Register File being REGFGP0. Any additional register files require the number after the
name of the register file. Register files with additional numbers after them are bank switched to reduce space, hence why the number is
required to denote the register file space used.
%\paragraph{}To alleviate context switching delays, three general purpose register files are bank availble by bank switching. Only these register files are required
%for a minimal system, though more can certainly be implemented if required. Each register file can be accessed by changing the value in the BSELRF register. 
\clearpage
\subsection{General Purpose Registers}
\paragraph{}32 general purpose registers that are 32 bits wide are available, as shown in Figure 1, in the previous section. 
There is a distinction between the System Register File and the General Purpose Register File, as during certain syscall
instructions, the register files are bank switched. Only GP0 through GP7 are saved, for passing between the different banks. 

\paragraph{}While it is not defined by the architecture, larger general purpose registers can be used instead of 32 bit wide registers. The System Register File allows 8 bit addressing for the registers to be accessed, in order to utilize more of the memory space.
If larger registers are needed, consider using a co-processor to for instructions that require larger operands. 
This provides code compatibility between different implementations.
\begin{figure}[!h]
	\caption{General Purpose Registers}
	%	\begin{center}
	\parbox{0.5\linewidth}{
		\centering
		\fontsize{10}{12}\selectfont
		\begin{tabular}{|l|l|}
			\hline
			\multicolumn{2}{|c|}{GPREGF} \\
			\hline
			Register & Register Name \\ \hline
			\$R0  & ZERO \\ 	\hline
			\$R1  & SP \\ 	\hline
			\$R2  & FP \\ 	\hline
			\$R3  & GP \\ 	\hline
			\$R4  & RA \\ 	\hline
			\$R5  & ARG0 \\ 	\hline
			\$R6  & ARG1 \\ 	\hline
			\$R7  & ARG2 \\ 	\hline
			\$R8  & ARG3 \\ 	\hline
			\$R9  & RVAL0 \\ 	\hline
			\$R10 & RVAL1 \\ \hline
			\$R11 & GR0  \\ \hline
			\$R12 & GR1  \\ \hline
			\$R13 & GR2  \\ \hline
			\$R14 & GR3  \\ \hline
			\$R15 & GR4  \\ \hline
			\$R16 & GR5  \\ \hline
			\$R17 & GR6  \\ \hline
			\$R18 & GR7  \\ \hline
			\$R19 & GR8  \\ \hline
			\$R20 & GR9  \\ \hline
			\$R21 & GR10  \\ \hline
			\$R22 & TMP0  \\ \hline
			\$R23 & TMP1 \\ \hline
			\$R24 & TMP2  \\ \hline
			\$R25 & TMP3  \\ \hline
			\$R26 & TMP4  \\ \hline
			\$R27 & TMP5  \\ \hline
			\$R28 & TMP6  \\ \hline
			\$R29 & TMP7  \\ \hline
			\$R30 & HI0  \\ \hline
			\$R31 & LOW0  \\ \hline
		\end{tabular}
	}
	\hfill
	\parbox{0.5\linewidth}{
		\centering
		\fontsize{10}{12}\selectfont
		\begin{tabular}{|l|l|}
			\hline
			\multicolumn{2}{|c|}{SYSREGF} \\
			\hline
			Register & Register Name \\ \hline
			\$R0  & ZERO \\ 	\hline
			\$R1  & SP1 \\ 	\hline
			\$R2  & FP1 \\ 	\hline
			\$R3  & GP1 \\ 	\hline
			\$R4  & RA1 \\ 	\hline
			\$R5  & SYSARG0 \\ 	\hline
			\$R6  & SYSARG1 \\ 	\hline
			\$R7  & SYSARG2 \\ 	\hline
			\$R8  & SYSARG3 \\ 	\hline
			\$R9  & SYSARG4 \\ 	\hline
			\$R10 & SYSARG5 \\ \hline
			\$R11 & SYSRVAL0  \\ \hline
			\$R12 & SYSRVAL1  \\ \hline
			\$R13 & SYSRVAL2  \\ \hline
			\$R14 & SYSRVAL3  \\ \hline
			\$R15 & SYSRVAL4  \\ \hline
			\$R16 & GPR0  \\ \hline
			\$R17 & GPR1  \\ \hline
			\$R18 & GPR2  \\ \hline
			\$R19 & GPR3  \\ \hline
			\$R20 & GPR4  \\ \hline
			\$R21 & GPR5  \\ \hline
			\$R22 & GPR6  \\ \hline
			\$R23 & GPR7  \\ \hline
			\$R24 & SYSTMPR0  \\ \hline
			\$R25 & SYSTMPR1  \\ \hline
			\$R26 & SYSTMPR2  \\ \hline
			\$R27 & SYSTMPR3  \\ \hline
			\$R28 & SYSTMPR4  \\ \hline
			\$R29 & SYSTMPR5  \\ \hline
			\$R30 & SYSREGHI0  \\ \hline
			\$R31 & SYSREGLOW0  \\ \hline
		\end{tabular}
	}
	%		\end{center}
\end{figure}

\clearpage


\subsection{Special Purpose Registers}
\paragraph{}The special registers are sorted between the System Registers and the Supervisor Registers. The system registers provide
simple configuration values and some read-only registers to give the programmer information about the implementation.
The registers defined in this manual are the bare minimum special purpose registers, and
should be included for code compatibility. \\
The supervisor registers are aimed at higher level functions required for operating system environments. They are not essential for operation,
and can either be partially implemented or not at all. The optional parts will be noted in the register descriptions.\\

\begin{figure}[!h]
	
	\caption{System Registers}
	\begin{center}
		\fontsize{10}{12}\selectfont
		\begin{tabular}{|l|l|l|l|}
			\hline
			\multicolumn{4}{|c|}{System Registers} \\
			\hline
			Register Name 	& Description 				 		& Width (bytes) 	& Address (Hex) \\ \hline
			CPUREV  		& CPU Revision 				 		& 	1				& 0x00000000 	\\ \hline
			CPNUM			& Co-Processor Number		 		&	1				& 0x00000001 	\\ \hline
			STAT			& Status Register			 		&	1				& 0x00000002 	\\ \hline
			n/a				& RESERVED				 	 		&	1				& 0x00000003 	\\ \hline
			OPCARP			& Opcode Registration Table Pointer	&	4				& 0x00000004 	\\ \hline
			CPIDTP 			& Co-Processor ID Table Pointer 	& 	4				& 0x00000008 	\\ \hline
			UINTEN			& User Interrupt Enable				&	4				& 0x0000000c	\\ \hline
			RPTINFO			& Running Process Thread Info		&	4				& 0x00000010 	\\ \hline
			RPID			& Running Process ID				&	2				&	0x00000012  \\ \hline

			
		\end{tabular}
	\end{center}
\end{figure}

\begin{figure} [!h]
	\caption{Supervisor Registers}
	
	\begin{center}
		\fontsize{10}{12}\selectfont
		\begin{tabular}{|l|l|l|}
			\hline
			\multicolumn{3}{|c|}{Supervisor Registers} \\
			\hline
			Register Name 	& Description 								& Address (Hex) \\ \hline
			SMSTAT 			& Supervisor mode status 					& 0x00000000 	\\ \hline
			RPCPR0 			& Running Process Pointer 					& 0x00000004	\\ \hline 
			RPITP			& Running Process Info Table Pointer		& 0x00000008	\\ \hline
			HTINFOP			& Hardware Thread Info Table Pointer		& 0x0000000c 	\\ \hline
			HTCTLP 			& Hardware Thread Control Table Pointer		& 0x00000010 	\\ \hline
			ECODE			& Exception Code							& 0x00000014	\\ \hline
			SYSCTP			& System Call Table Pointer					& 0x00000015	\\ \hline
			n/a				& RESERVED									& 0x00000018	\\ \hline
		\end{tabular}
	\end{center}
	
\end{figure}
\clearpage
\subsection{Special Purpose Register Definitions}

\subsubsection{System Registers}

% Register Template 8 bits
%REGNAME register START
%\begin{center}
%	\begin{tabular}{|c|c|c|c|c|c|c|c|}
%		\multicolumn{8}{c}{\textbf{}:  Register} \\ \hline
%		7 & 6  &  5   &	4	 &	3	&	2  &	1 &	0 \\ \hline
%		&  &  &  &  &  &  &   \\ \hline
		
%	\end{tabular}
%\end{center}


%\begin{center}
%	\line(1,0){250}
%\end{center}

%REGNAME register END






%CPUREV register START
\begin{center}
	\begin{tabular}{|cccc|cccc|}
		\multicolumn{8}{c}{\textbf{CPUREV}: CPU Revision Register} \\ \hline
		7 \hfill &\hfill & \hfill&\hfill4\hfill &\hfill3\hfill &\hfill &\hfill &\hfill0 \\ \hline
		\multicolumn{4}{|c|}{Major CPU Revision} & \multicolumn{4}{|c|}{ Minor CPU Revision}  \\ \hline
		
	\end{tabular}
\end{center}

\paragraph{} The \textbf{CPUREV} register holds the revision number for the processor
implementation. The upper 4 bits hold the major revision number, the lower 4 bits the 
minor revision number. This number is hard coded by the implementation. The
major revision number should refer to the company implementation, with the minor
revision number is for the implementation's revision.
\paragraph{} Any permission level can read this register. Since it should be hard coded,
there is no way to write to this register.

\begin{center}
	\line(1,0){250}
\end{center}
\vspace{0.2 cm}
%CPUREV register END


%CPNUM register START
\begin{center}
	\begin{tabular}{|cccccccc|}
		\multicolumn{8}{c}{\textbf{CPNUM}: Co-Processor Number Register} \\ \hline
		7 & \hfill  & \hfill & \hfill & \hfill & \hfill & \hfill &\hfill	0 \\ \hline
		\multicolumn{8}{|c|}{Number of Co-processors}   \\ \hline

	\end{tabular}
\end{center}

\paragraph{} The \textbf{CPNUM} register simply holds the number of different types of
co-processors in the processor implementation. This register is used for finding out how many CPIDs to check, for determining whether to use microcode or the co-processor
hardware available for each co-processor.
\paragraph{} This register can be read by with any permission level. This register
should be hard coded, so no writes are possible.

\begin{center}
	\line(1,0){250}
\end{center}
\vspace{0.2 cm}
%CPNUM register END




\clearpage

%STAT register START
\begin{center}
	\begin{tabular}{|c|c|c|c|c|c|c|c|}
	%	\hline
			\multicolumn{8}{c}{\textbf{STAT}: Status Register} \\ \hline
		 	7 & 6  &  5   &	4	 &	3	&	2  &	1 &	0 \\ \hline
			Z & OV & PEMA & PML2 & PML1 & PML0 & INTN & SPCP  \\ \hline
		
	\end{tabular}
\end{center}

\begin{tabular}{l l}
	STAT	& Read only register for various processor state information. The flags are \\ 
	\hfill 	& explained in detail below. \\ \hline
	\hfill	&	\hfill \\ 
	Z		& Zero Flag; Indicates whether the processed instruction's resulted in zero. \\
	\hfill	& Read Only. The flag is set to 0 when the ALU calculation is 1, and 0 when \\
	\hfill &  the output is non-zero. \\
	OV		& Overflow Flag; Indicated whether the processed instruction's result \\
	\hfill	& overflowed the 32 bit space. Read only. \\
	\hfill	& The overflow flag is set to the output of the carry out of the ALU. \\
	\hfill  & With addition, this would set the bit to a 1 when true. \\
	\hfill	& For subtraction, the inverse is true. \\ 
	PML 	& Permission Level; Inidcates the running process' permission level. Read only.\\
	\hfill	&	\begin{tabular}{l|l}
			PML & Description \\ \hline
			000 & Low User Level \\
			001 & High User Level \\
			010 & Low Supervisor Level \\
			011 & High Supervisor Level \\
			100 & Low Hypervisor Level \\
			101 & High Hypervisor Level \\
			110 & Reserved \\
			111 & Top Hypervisor Level  \\
	\end{tabular}\\
	\hfill & The permission levels are explained in more detailed in their dedicated section.\\
	PEMA   & Permission Accepted; Indictates whether a privaledged system call  \\
	\hfill & request was accepted. Read only \\
	GINE   & Global Interrupt Enable; Indicates whether interrupts are enabled. Read/Write \\ 
	SPCP   & Support CoProcessors; Indicates whether co-processor code is recognized as \\
	\hfill & illegal instruction or is handled by an interrupt. (0 is illegal, 1 is  \\
	\hfill & interrupt handled) Read/Write. \\

\end{tabular}
\paragraph{} The \textbf{STAT} register provides the programmer with various information about the current state of the processor.
\paragraph{} In order to write to this register the bits that are defined as read only should be set to 0.  \\
\paragraph{Optional} In the event that the permission levels are not needed, they should be hard coded to 0x7, the
highest permission level to avoid porting code. PEMA should also be hardcoded to a logic high for the same reasons stated.\\

\begin{center}
	\line(1,0){250}
\end{center}
\vspace{0.2 cm}
%STAT register END

\clearpage

%OPCARP register START
\begin{center}
	\begin{tabular}{|cccccccc|}
		\multicolumn{8}{c}{\textbf{OPCARP}: Opcode Registration Table Pointer  Register} \\ \hline
		31\hfill & \hfill  & \hfill & \hfill & \hfill & \hfill & \hfill &\hfill	0 \\ \hline
		\multicolumn{8}{|c|}{Address}   \\ \hline

	\end{tabular}
\end{center}

\paragraph{} The \textbf{OPCARP} register is a pointer for the Opcode Registration Table.
This table exists to map co-processor opcodes to available opcodes in the processor.
It is not necessary for the table to be hard coded, and this option is left up
to the implementation.
\paragraph{} Only permission levels higher than or equal to the High Supervisor Level
can read or write to this register. If this register is not configured correctly
before running an Operating System or bare-metal program, unintentional CPMI interrupts
may occur with no proper service routine.

\begin{center}
	\line(1,0){250}
\end{center}

%OPCARP register END
\vspace{1 cm}

%CPIDTP register START
\begin{center}
	\begin{tabular}{|cccccccc|}
		\multicolumn{8}{c}{\textbf{CPIDTP}: Co-Processor ID Table Pointer  Register} \\ \hline
		31\hfill & \hfill  & \hfill & \hfill & \hfill & \hfill & \hfill &\hfill	0 \\ \hline
		\multicolumn{8}{|c|}{Address}   \\ \hline
		
	\end{tabular}
\end{center}

\paragraph{} The \textbf{CPIDTP} register is a pointer for the Co-Processor ID Table.
This table holds all co-processor IDs to allow the programmer to know how to handle software implementations of various co-processors.
\paragraph{} All permission levels can read from the CPIDTP register. Only permission
levels higher than the High Supervisor Level can write to this register, if it
has not been hard coded.

\begin{center}
	\line(1,0){250}
\end{center}

%CPIDTP register END
\vspace{1 cm}



%UINTEN register START
\begin{center}
	\begin{tabular}{|c|c|c|c|c|c|c|c|c|c|c|c|c|c|c|c|}
	%	\hline
			\multicolumn{16}{c}{\textbf{UINTEN}: Interrupt Enable Register}  \\ \hline

		 	31 	& 30  	&  29   &	28	&	27	&	26 	&   25 		&	24 & 23 	& 22  	&  21  	&	20	&	19	&	18 	&   17 		&	16 		\\ 
			\hspace{5pt} 	& \hspace{5pt} 	& \hspace{5pt} 	& \hspace{5pt} 	& \hspace{5pt}	& \hspace{5pt} 	& \hspace{5pt}	& \hspace{5pt}   &
			\hspace{5pt} 	& \hspace{5pt} 	& \hspace{5pt} 	& \hspace{5pt} 	& \hspace{5pt}	& \hspace{5pt} 	& \hspace{5pt}	& \hspace{5pt}    \\ \hline
		 	15 	& 14  	&  13   &	12	&	11	&	10 	&	9 		&	8 & 7 	& 6  	&  5   	&	4	&	3	&	2  	&	1 		&	0 		\\ 
			NA 	& NA 	& NA 	& NA 	& NA 	& NA 	& NA 		& NA   &
			NA 	& NA 	& NA 	& NA 	& NA 	& NA   	& ROI  		& CPMI    \\ \hline
		
	\end{tabular}
\end{center}

\paragraph{} The \textbf{UINTEN} register holds the settings for enabling various interrupts. More information about each interrupt
is available in the interrupt section. The spots that are blank are usable by the implementation, while spots marked 'NA'
are reserved for future use. Setting a bit field to 1 enables the interrupt vector, after ensuring that the GINE bit in \textbf{STAT} is set.

\begin{center}
	\line(1,0){250}
\end{center}
\vspace{0.2 cm}
%UINTEN register END
\clearpage


%RPTINFO register START
\begin{center}
	\begin{tabular}{|cc|cc|cccc|}
		\multicolumn{8}{c}{\textbf{RPTINFO}: Running Process Thread Info Register} \\ \hline
		31\hfill & 16\hfill  & 15\hfill & \hfill 8 & 7 \hfill & \hfill & \hfill &\hfill	0 \\ \hline
		 \multicolumn{2}{|c|}{PPID} & \multicolumn{2}{|c|}{CHPID} & \multicolumn{4}{|c|}{RESERVED} \\ \hline
		
	\end{tabular}
\end{center}

\paragraph{} The \textbf{RPTINFO} register provides information about a process's threads. This information includes information for parent and child processes. The register fields are described in detail below.

\begin{tabular}{l l}

	PPID		& Parent Process ID. Holds the parent's process ID. \\
	CHPID		& Child Process ID. The current thread number is stored here. \\
	\hfill		& 256 total threads are currently possible with using this scheme. \\
	RESERVED	& Reserved for future use. \\
\end{tabular}
\paragraph{} All permission levels can read this register. Only permission levels equal to or higher than the High Supervisor Levels can write to this register. 
\begin{center}
	\line(1,0){250}
\end{center}
\vspace{1cm}
%RPTINFO register END

%RPID register START
\begin{center}
	\begin{tabular}{|cccccccc|}
		\multicolumn{8}{c}{\textbf{RPID}: Running Process ID  Register} \\ \hline
		16 & \hfill & \hfill & \hfill &	\hfill & \hfill & \hfill & \hfill0 \\ \hline
		\multicolumn{8}{|c|}{ID number}  \\ \hline

	\end{tabular}
\end{center}

\paragraph{} The RPID register holds the running process' ID number. This register
is to be set by the kernel, and is not necessary otherwise. All permission levels
can read from the register. Only permission levels equal to or higher than the High Supervisor Levels can write to this register. 

\begin{center}
	\line(1,0){250}
\end{center}

%RPID register END



\clearpage
\subsubsection{Supervisor Registers}
\paragraph{} This section is still being developed. It will be updated in the future.

\clearpage
\section{Permission Levels}
\paragraph{} The various permission levels are designed to allow for protected execution between user space, a kernel, and virtual
environments. Permission levels are changed through dedicated instructions to increase or decrease the permission level counter. System calls
can change the permission level in order to execute specific functions, but can be revoked in order to keep security. The individual function
calls can be defined in the System Call Function Pointer Table (SYSFPT). 
\paragraph{} On boot, the default permission level is initialized to the highest permission level in order to properly configure the processor.
There are no restrictions to access. However, after configuration the permission level should be lowered as required to ensure secure execution.
\subsection{User Levels}

\subsubsection{Low User Level} User space processes are designed to run in this permission level. Write access to supervisor registers is revoked,
and all modifying system calls are checked for allowed execution. Low priority, non time sensitive applications, high security risk applications
should be run at this level. Since the Low User Level is the most restrictive priority level, in the event of a illegal operation by the application
the processor should attempt to revert to a state before the illegal action and either kill or halt the offending process in order to prevent further
illegal actions. Halting the process should be used to determine more information about the illegal action, while killing the process is ideal for
more critical situations. 

\subsubsection{High User Level} This user level is designed to keep some security between user space and supervisor space, whiling loosening
restrictions. Only reads are allowed to system and supervisor registers, and a limited number of system calls will not need to be verified. 
The verified system calls are specified in the 8 system call registers: SCPMHULX. Time critical user space processes, trusted system call 
saturated processes, and processes that require reading system and supervisor registers should use this permission level. 
registers. 

\subsection{Supervisor Levels}
\subsubsection{Low Supervisor Level} This permission level allows reads and writes to most system registers, with few restrictions. Hardware drivers,
real time processes, and kernel tasks that do not require low level configuration should use this level.
\subsubsection{High Supervisor Level} This permission level is for unrestricted access for operating system environments. Essential Kernel processes and
bare-metal embedded applications are examples for using this permission level. All configuration registers are available, only hypervisor configuration and
actions are restricted. If the hypervisor system is not implemented, this level is the highest permission level.

\subsection{Hypervisor Levels}
\paragraph{} This section is still under development. At this time, the Hypervisor levels simply allow access to all functions of a processor, including the
virtualization features. Since the virtualization functions are not defined yet, this is a placeholder for the time being.

%\subsubsection{Low Hypervisor Level}
%\subsubsection{High Hypervisor Level}
%\subsubsection{Top Hypervisor Level}
\clearpage
\section{Memory}
\paragraph{}At this point in time, the ISA only handles 32 bit addresses.
With memory capacity increasing in size as time goes on, this may change. %In order to access 
%larger sections of memory, the use of an offset register may be implemented to create 64 bit addresses.
%The use of virtual memory with said offset register is also accepted.
\paragraph{} The Fusion-ISA uses a modified Harvard architecture, such that the programmer should have the appearance of a Harvard architecture.
Compliance to this is not strict, as any variation in between is allowed, but the separation between data and instruction memory should be made.

\subsection{Memory Locations for Vector Table}
\subsubsection{Interrupt Vector Table}
\paragraph{} The table below is the definition for the Interrupt Vectors. The vectors are padded by two addresses
in order to give enough space for a call macro, which expands to two instructions. Additional interrupt vectors 
can be created, though the ones specified are necessary.

\begin{figure}[!h]
	\begin{center}
		\begin{tabular}{|l|l|l|}
			\hline
			Address (32 bit) 	& Name  & Definition						\\  \hline
			0x0000 				& Reset & Reset processor					\\ 	\hline
			0x0008 				& CPMI  & Co-Processor Microcode Interrupt 	\\ 	\hline
			0x0010 				& ROI	& Register Overflow (Arithmetic)	\\ 	\hline
			0x0018				&  RESERVED&\\ 	\hline
			0x0020 				& RESERVED &\\ 	\hline
			0x0028 				& RESERVED &\\	\hline
		\end{tabular} 
		\caption{Interrupt Vector Table}
	\end{center}
	
\end{figure} 

\begin{center}
	\line(1,0){250}
\end{center}

\subsubsection{Exception List}

\begin{figure}[!h]
	\begin{center}
		\begin{tabular}{|l|l|l|}
			\hline
			Exception Code 		& Name		& Definition						\\ \hline
			0x0000 				& INSFPEM	& Insufficient Permissions 			\\ \hline
			0x0001 				& MISADDR	& Misaligned Address	 			\\ \hline
			0x0002 				& HALT		& Halt Request			 			\\ \hline
			0x0003 				& PTERM		& Terminate Process		 			\\ \hline
			0x0004 				& PKILL		& Kill Process			 			\\ \hline
			0x0005				& PEXIT		& Process Exit (End)				\\ \hline
			0x0006				& TTERM		& Terminate Thread					\\ \hline
			0x0007				& TKILL		& Kill Thread						\\ \hline
			0x0008				& TEXIT		& Exit Thread (End)					\\ \hline
			0x0009				& INVSYS	& Invalid System Call				\\ \hline
			0x000a				& INVPR		& Invalid Permission Request		\\ \hline
		\end{tabular} 
		\caption{Exception List}
	\end{center}
	
\end{figure} 


%\begin{center}
%	\line(1,0){250}
%\end{center}

\clearpage
\section{Instruction Usage}
\paragraph{}This section will go over programming use of the instructions including the affects on the processor. Information on the 
instruction encodings and list of binary representations of each instruction is available in the 'Instructions' Section.
\subsection{Integer}
\paragraph{} The following instructions operate on two registers, RSa and RSb, and store the result into Rd. The Z and OV flags are affected
by the output of the instructions listed below. Overflows of register operations can be handled by the program, or the processor by checking the
OV flag in the \textbf{STAT} register, or enabling the ROI interrupt. The Z flag is available as well to the programmer, but it does not have
it's own interrupt to use. Since branch instructions exist to check if a register is equal to zero, the Z flag does not serve a purpose at this time.
\paragraph{} Since arithmetic functions are simple to explain, the following table will denote the operands and function of the Integer instructions. 

\begin{figure}[!h]
	\begin{center}
		\begin{tabular}{|l|l|l|}
			\hline
			\multicolumn{3}{c}{Integer Instructions} \\ \hline
			Instruction Name	& Definition															& Usage 							\\ \hline
			add					& Adds RSa and RSb, result stored in Rd									& \texttt{add \$Rd, \$RSa, \$RSb}	\\ \hline
			sub					& Subtracts RSb from RSa, result stored in Rd							& \texttt{sub \$Rd, \$RSa, \$RSb}	\\ \hline
			addu 				& Adds RSa and RSb, result stored in Rd 								& \texttt{addu \$Rd, \$RSa, \$RSb}	\\ 
			\hfill				& Does not affect OV flag												& \hfill 							\\ \hline
			subu 				& Subtracts RSb from RSa, result stored in Rd							& \texttt{subu \$Rd, \$RSa, \$RSb}	\\ 
			\hfill				& Does not affect OV flag												& \hfill 							\\ \hline
			not					& Logically inverts RSa, result stored in Rd							& \texttt{not \$Rd, \$RSa} 			\\ \hline
			and					& Logic AND operation on RSa and RSb, result							& \texttt{and \$Rd, \$RSa, \$RSb}	\\	
			\hfill				& stored in Rd															& \hfill 							\\ \hline
			or					& Logic OR operation on RSa and RSb, 									& \texttt{or \$Rd, \$RSa, \$RSb	}	\\ 	
			\hfill				& result stored in Rd													& \hfill 							\\ \hline
			xor					& Logic XOR operation on RSa and RSb,									& \texttt{xor \$Rd, \$RSa, \$RSb}	\\ 
			\hfill				& result stored in Rd													& \hfill 							\\ \hline	
			sal					& Arithmetic Shift RSa Left by the value of RSb,						& \texttt{sal \$Rd, \$RSa, \$RSb}	\\ 	
			\hfill				& result stored in Rd													& \hfill 							\\ \hline
			sar					& Arithmetic Shift RSa Right by the value of RSb,						& \texttt{sar \$Rd, \$RSa, \$RSb}	\\ 	
			\hfill				& result stored in Rd													& \hfill 							\\ \hline
			sll					& Logic Shift RSa Left by the value of RSb,								& \texttt{sll \$Rd, \$RSa, \$RSb}	\\ 
			\hfill				& result stored in Rd													& \hfill 							\\ \hline	
			slr					& Logic Shift RSa Right by the value of RSb, 							& \texttt{slr \$Rd, \$RSa, \$RSb}	\\ 	
			\hfill				& result stored in Rd													& \hfill 							\\ \hline
			comp				& Compares the values of RSa and RSb. 									& \texttt{comp \$Rd, \$RSa, \$RSb}	\\
			\hfill				& Sets Rd to 0x0 if equal,												& \hfill							\\
			\hfill				& 1 if RSa is greater than RSb, 										&	\hfill							\\ 
			\hfill				& and -1 if RSa is less than RSb										& \hfill							\\ \hline
		\end{tabular} 
	\end{center}
	
\end{figure} 

%\begin{center}
%	\line(1,0){250}
%\end{center}

\clearpage
\subsection{Immediate}
\paragraph{} Similar to the Integer instructions, the Immediate instructions perform arithmetic operations on the RSa register and a 12 bit immediate.
The table shown below describes the instructions available.



\begin{figure}[!h]
	\begin{center}
		\begin{tabular}{|l|l|l|}
			\hline
			\multicolumn{3}{c}{Immediate Instructions} \\ \hline
			Instruction Name	& Definition															& Usage 							\\ \hline
			addi				& Adds RSa and Immediate, result stored in Rd							& \texttt{addi \$Rd, \$RSa, 0xfff}	\\ \hline
			subi				& Subtracts Immediate from RSa, result stored in Rd						& \texttt{subi \$Rd, \$RSa, 0xfff}	\\ \hline
			addui 				& Adds RSa and Immediate, result stored in Rd 							& \texttt{addui \$Rd, \$RSa, 0xfff}	\\ 
			\hfill				& Does not affect OV flag												& \hfill 							\\ \hline
			subui 				& Subtracts Immediate from RSa, result stored in Rd						& \texttt{subui \$Rd, \$RSa, 0xfff}	\\ 
			\hfill				& Does not affect OV flag												& \hfill 							\\ \hline
			noti				& Logically inverts Immediate, result stored in Rd						& \texttt{noti \$Rd, 	0xfff}		\\ \hline
			andi				& Logic AND operation on RSa and Immediate, 							& \texttt{andi \$Rd, \$RSa, 0xfff}	\\ 
			\hfill				& result stored in Rd													& \hfill 							\\ \hline	
			ori					& Logic OR operation on RSa and Immediate, 								& \texttt{ori \$Rd, \$RSa, 0xfff}	\\ 
			\hfill				& result stored in Rd													& \hfill 							\\ \hline		
			xori				& Logic XOR operation on RSa and Immediate, 							& \texttt{xori \$Rd, \$RSa, 0xfff}	\\ 
			\hfill				& result stored in Rd													& \hfill 							\\ \hline					
			sali				& Arithmetic Shift RSa Left by the Immediate value,						& \texttt{sali \$Rd, \$RSa, 0xfff}	\\ 	
			\hfill				& result stored in Rd													& \hfill 							\\ \hline				
			sari				& Arithmetic Shift RSa Right by the Immediate value,					& \texttt{sari \$Rd, \$RSa, 0xfff}	\\ 	
			\hfill				& result stored in Rd													& \hfill 							\\ \hline				
			slli				& Logic Shift RSa Left by the Immediate value, 							& \texttt{slli \$Rd, \$RSa, 0xfff}	\\ 	
			\hfill				& result stored in Rd													& \hfill 							\\ \hline				
			slri				& Logic Shift RSa Right by the Immediate value, 						& \texttt{slri \$Rd, \$RSa, 0xfff}	\\ 	
			\hfill				& result stored in Rd													& \hfill 							\\ \hline				
			compi				& Compares the RSa and the Immediate value. 							& \texttt{compi \$Rd, \$RSa, 0xfff}	\\
			\hfill				& Sets Rd to 0x0 if equal, 1 if RSa is greater than the					& \hfill							\\
			\hfill				& Immediate value, and -1 if RSa is less								& \hfill							\\
			\hfill				& than the Immediate value												& \hfill 							\\ \hline
		\end{tabular} 
	\end{center}
	
\end{figure}


\begin{center}
	\line(1,0){250}
\end{center}
\clearpage
\subsection{Load/Store}
\paragraph{} The Load Instructions feature a 14 bit immediate offset value, with a base address of the value of RSa. Since the architecture is big endian,
the byte aligned address refers to the most significant byte. When reading values less than 32 bits wide, the upper bytes are set to zeros. For example, when loading
a two byte value into Rd, the address in RSa and the next byte will be placed into the lower two bytes of Rd.
\begin{figure}[!h]
	\begin{center}
		\begin{tabular}{|l|l|l|}
			\hline
			\multicolumn{3}{c}{Load Instructions} \\ \hline
			Instruction Name	& Definition											& Usage 							\\ \hline
			lw					& Reads a 4 byte value from Memory at address RSa 		& \texttt{lw \$Rd, 0x3fff(\$RSa)}	\\ 
			\hfill				& with a 14 bit Immediate offset, puts result in Rd		& \hfill 							\\ \hline
			lh					& Reads a 2 byte value from Memory at address RSa 		& \texttt{lw \$Rd, 0x3fff(\$RSa)}	\\ 
			\hfill				& with a 14 bit Immediate offset, puts result in Rd		& \hfill 							\\ \hline			
			lb					& Reads a 1 byte value from Memory at address RSa 		& \texttt{lw \$Rd, 0x3fff(\$RSa)}	\\ 
			\hfill				& with a 14 bit Immediate offset, puts result in Rd		& \hfill 							\\ \hline			
			lth					& Reads a 3 byte value from Memory at address RSa 		& \texttt{lw \$Rd, 0x3fff(\$RSa)}	\\ 
			\hfill				& with a 14 bit Immediate offset, puts result in Rd		& \hfill 							\\ \hline			
		
		\end{tabular} 
	\end{center}
	
\end{figure}

\begin{center}
	\line(1,0){250}
\end{center}

\paragraph{} The Store Instructions are similar in nature to the Load instructions, with a 14 bit immediate value offset to the base address of \$RSa.
The value of \$RSb will be written to the address calculated from the aforementioned sources. For writing values smaller than 32 bits wide, the lower
bytes will be written in a big endian fashion to memory. For example, writing a two byte value to memory, would take the lower two bytes of \$RSb, and write
the higher byte to the address calculated and the lower byte to the address calculated plus 1. 
\begin{figure}[!h]
	\begin{center}
		\begin{tabular}{|l|l|l|}
			\hline
			\multicolumn{3}{c}{Store Instructions} \\ \hline
			Instruction Name	& Definition											& Usage 							\\ \hline
			sw					& Writes the value of RSb at address RSa				& \texttt{lw \$RSb, 0x3fff(\$RSa)}	\\ 
			\hfill				& with a 14 bit Immediate offset.						& \hfill 							\\ \hline
			sh					& Writes the lower two bytes of RSb 					& \texttt{lw \$RSb, 0x3fff(\$RSa)}	\\ 
			\hfill				& at address RSa with a 14 bit Immediate offset.		& \hfill 							\\ \hline
			sb					& Writes the least significant byte of RSb 				& \texttt{lw \$RSb, 0x3fff(\$RSa)}	\\ 
			\hfill				& at address RSa with a 14 bit Immediate offset.		& \hfill 							\\ \hline
			sth					& Writes the lower three bytes of RSb 					& \texttt{lw \$RSb, 0x3fff(\$RSa)}	\\ 
			\hfill				& at address RSa with a 14 bit Immediate offset.		& \hfill 							\\ \hline
		 		\end{tabular} 
	\end{center}
	
\end{figure}

\begin{center}
	\line(1,0){250}
\end{center}
\clearpage
\paragraph{} The Load Immediate Instructions are for putting immediate values into registers. A 16 bit immediate value is stored into Rd, with variations on
register file and location within the register.
\begin{figure}[!h]
	\begin{center}
		\begin{tabular}{|l|l|l|}
			\hline
			\multicolumn{3}{c}{Load Immediate Instructions} \\ \hline
			Instruction Name	& Definition								& Usage 						\\ \hline
			li					& Put Immediate value into the 				& \texttt{li \$Rd, 0xffff}		\\ 
			\hfill				& lower two bytes of Rd in the GPREGF		& \hfill						\\ \hline
			lui					& Put Immediate value into the  			& \texttt{lui \$Rd, 0xffff}		\\ 	
			\hfill				& upper two bytes of Rd in the GPREGF		& \hfill 						\\ \hline
			lni					& Put unsigned Immediate value into the		& \texttt{lni \$Rd, 0xffff}		\\ 
			\hfill				& lower two bytes of Rd in the GPREGF		& \hfill						\\ \hline
			luni				& Put unsigned Immediate value into the		& \texttt{luni \$Rd, 0xffff}	\\ 
			\hfill				& upper two bytes of Rd in the GPREGF		& \hfill 						\\ \hline

			lgi					& Put Immediate value into the 				& \texttt{lgi \$Rd, 0xffff}		\\ 
			\hfill				& lower two bytes of Rd in the GLBREGF		& \hfill						\\ \hline
			lugi				& Put Immediate value into the 				& \texttt{lugi \$Rd, 0xffff}	\\ 
			\hfill				& upper two bytes of Rd in the GLBREGF		& \hfill						\\ \hline
			lngi				& Put unsigned Immediate value into the		& \texttt{lngi \$Rd, 0xffff}	\\ 
			\hfill				& lower two bytes of Rd in the GLBREGF		& \hfill						\\ \hline
			lungi				& Put unsigned Immediate value into the 	& \texttt{lungi \$Rd, 0xffff}	\\ 
			\hfill				& upper two bytes of Rd in the GLBREGF		& \hfill						\\ \hline


			lsi					& Put Immediate value into the 				& \texttt{lsi \$Rd, 0xffff}		\\ 
			\hfill				& lower two bytes of Rd in the SYSREGF		& \hfill 						\\ \hline
			lusi				& Put Immediate value into the 				& \texttt{lusi \$Rd, 0xffff}	\\ 
			\hfill				& upper two bytes of Rd in the SYSREGF		& \hfill						\\ \hline
			lnsi				& Put unsigned Immediate value into the 	& \texttt{lnsi \$Rd, 0xffff}	\\ 
			\hfill				& lower two bytes of Rd in the SYSREGF		& \hfill 						\\ \hline
			lunsi				& Put unsigned Immediate value into the 	& \texttt{lunsi \$Rd, 0xffff}	\\ 
			\hfill				& upper two bytes of Rd in the SYSREGF		& \hfill						\\ \hline			
		\end{tabular} 
	\end{center}
	
\end{figure}

\begin{center}
	\line(1,0){250}
\end{center}
 
 \newpage
\subsection{Branch/Jump}
\paragraph{} The Branch Instructions conditionally change the Program Counter based off of the operation of RSa and RSb. The 14 bit
immediate field is PC relative to the address of the branch instruction.

\begin{figure}[!h]
	\begin{center}
		\begin{tabular}{|l|l|l|}
			\hline
			\multicolumn{3}{|c|}{Branch Instructions} \\ \hline
			Instruction Name	& Definition									& Usage 							\\ \hline
			beq					& Branch to PC relative Immediate address		& \texttt{beq \$RSa, \$RSb, 0x3fff}	\\ 
			\hfill				& when RSa and RSb are equal					& \hfill 							\\ \hline
			bne					& Branch to PC relative Immediate address		& \texttt{bne \$RSa, \$RSb, 0x3fff}	\\ 
			\hfill				& when RSa and RSb are not equal				& \hfill 							\\ \hline
			bgt					& Branch to PC relative Immediate address		& \texttt{bgt \$RSa, \$RSb, 0x3fff}	\\ 
			\hfill				& when RSa is greater than RSb					& \hfill 							\\ \hline
			blt					& Branch to PC relative Immediate address		& \texttt{blt \$RSa, \$RSb, 0x3fff}	\\ 
			\hfill				& when RSa is less than RSb						& \hfill 							\\ \hline
		 		\end{tabular} 
	\end{center}
	
\end{figure}
\begin{center}
	\line(1,0){250}
\end{center}



\paragraph{} The Jump Instructions change the Program Counter with a 21 bit immediate relative offset. The immediate is relative to the address
of the jump instruction. Jump register instructions are absolute, and do not have a relative immediate. The value of RSa is used as the base address.
Jump instructions that link save the return address into RA. The return address is the next 4 byte aligned address after the jump instruction.

\begin{figure}[!h]
	\begin{center}
		\begin{tabular}{|l|l|l|}
			\hline
			\multicolumn{3}{|c|}{Jump Instructions} \\ \hline
			Instruction Name	& Definition										& Usage 							\\ \hline
			j					& PC relative jump									& \texttt{j 0x1fffff}	\\ \hline
			jal					& PC relative jump, save return address 			& \texttt{jal	0x1fffff}	\\ \hline
			jr					& Jump register instruction, address calculated by	& \texttt{jr 0x1fffff(\$RSa)}	\\
			\hfill				& immediate field plus RSa, save return address		&		\hfill				\\ \hline
			jrl					& Jump register instruction, address calculated by	& \texttt{jrl 0x1fffff(\$RSa)}	\\ 
			\hfill				& immediate field plus RSa, save return address		&		\hfill				\\ \hline
		 		\end{tabular} 
	\end{center}
	
\end{figure}
\begin{center}
	\line(1,0){250}
\end{center}

\newpage

\subsection{System}
\paragraph{} The system instructions are more complex, but designed to utilize similar features to the rest of the main core or simplistic
implementations of quick control sequences. Each instruction will get it's own paragraph due to the nature of the instructions
not having a lot of similarities between them.


\begin{figure}[!h]
	\begin{center}
		\begin{tabular}{|l|l|}
			\hline
			Name		& Usage \\ \hline
			syscall		& \texttt{syscall 0xff}	\\ \hline
			sysret		& \texttt{sysret}	\\ \hline
		 		\end{tabular} 
	\end{center}
	
\end{figure}
\paragraph{} The system call and system return instructions are used for their named purpose. The number used for the system calls are created
in the System Call Table, which defines the address for system call routines. The table pointer should be created before a system call is made,
otherwise a Invalid System Call exception will be raised. Permissions can be set for various system calls within the System Call Table, and the
programmer should ensure that the permission level is adequate for using a system call. If permissions are adequate, the permission level will
be raised appropriately and start executing the system call routine.
\paragraph{}The sysret instruction should be placed at the end of the system call routine in order to switch back
to the user space register file as well as lowering permissions.

\begin{center}
	\line(1,0){250}
\end{center}


\begin{figure}[!h]
	\begin{center}
		\begin{tabular}{|l|l|}
			\hline
			Name		& Usage \\ \hline
			stspr		& \texttt{stspr \$RSa, SPR}	\\ \hline
			ldspr		& \texttt{ldspr \$Rd, SPR}	\\ \hline
		 		\end{tabular} 
	\end{center}
	
\end{figure}
\paragraph{} These instructions are for writing and reading to special purpose registers, respectively. The special purpose register address is put
into the immediate field. Depending on the permission level setting of the registers, an Invalid Permission Level exception may be raised.
Refer to the specific registers to determine if the permission level is adequate.

\begin{center}
	\line(1,0){250}
\end{center}


\begin{figure}[!h]
	\begin{center}
		\begin{tabular}{|l|l|}
			\hline
			Name		& Usage \\ \hline
			sync		& \texttt{sync}	\\ \hline
		 		\end{tabular} 
	\end{center}
	
\end{figure}
\paragraph{} The sync instruction is for ensuring memory synchronization. After the sync function is executed, the preceding instructions
finish executing and all pending memory accesses to finish. The result allows a clean transition for context switches.

\begin{center}
	\line(1,0){250}
\end{center}

\begin{figure}[!h]
	\begin{center}
		\begin{tabular}{|l|l|}
			\hline
			Name		& Usage \\ \hline
			lock		& \texttt{lock \$Rd, \$RSa}	\\ \hline
			test		& \texttt{test \$Rd, \$RSa}	\\ \hline
		 		\end{tabular} 
	\end{center}
	
\end{figure}
\paragraph{} The test and lock instructions provide exclusivity in multi-processor configurations, and between co-processors. The
value in RSa is the address of the resource or memory location, and the success of the action is placed in Rd. If successful,
the result will be zero. If unsuccessful, the resource that was attempted to be accessed will put an error code within Rd.
The error code depends on the resource, and should be consulted in the respect location in the documentation of the co-processor or
various other resource.

\begin{center}
	\line(1,0){250}
\end{center}

\begin{figure}[!h]
	\begin{center}
		\begin{tabular}{|l|l|}
			\hline
			Name		& Usage \\ \hline
			pmir	& \texttt{pmir	0xff}	\\ \hline
			pmd		& \texttt{test	0xff}	\\ \hline
		 		\end{tabular} 
	\end{center}
	
\end{figure}
\paragraph{} These instructions are for increasing or decreasing the current permission level. The immediate field
denotes the permission level that the processor should change to. In the event that a permission level cannot be changed,
an Invalid Permission Level exception will be raised. Depending on the severity of the infraction, the running program
may be terminated or halted. More information will be available in the permission level section.

\begin{center}
	\line(1,0){250}
\end{center}


\subsection{Macros}
\paragraph{} The macro instructions defined here explain the usage and their expansions.

\begin{figure}[!h]
	\begin{center}
		\begin{tabular}{|l|l|l|}
			\hline
			Name		& Usage & Number of Instructions\\ \hline
			call	& \texttt{call	address\_label}		&	2	\\ \hline
			ret		& \texttt{ret	}					&	1	\\ \hline
			la		& \texttt{la \$Rd, address\_label}	& 	2	\\ \hline
		 		\end{tabular} 
	\end{center}
	
\end{figure}
\paragraph{} The call macro expands to a lui instruction and a jrl instruction with an immediate offset. The lower two bytes
are used as the offset for the jump register, while the lui instruction loads the upper two bytes into register TMP0. 
TMP0 should be ensured to be cleared before using a call instruction, or an undefined sequence of events may occur.
This register choice may change in the future to ensure better compatibility and resource usage, however register TMP0 should
be avoided in use around call instructions as it may cause unintentional side effects. 

\paragraph{} The ret macro allows the program to return to the program from the subroutine. It expands into a jr instruction
with no immediate offset. The RA register stores the return address, which the jr instruction uses. 

\paragraph{} The la macro loads the upper and lower bytes into the specified register, Rd. It expands into two instructions,
li and lui. Currently there is no way to merge the two values into one register, however the macro will be updated soon to do so.
It is advised against using this macro until further notice. The same functionality can be achieved with an or instruction and
a li and lui instruction, however due to register usage this is not ideal to create as a macro.
\begin{center}
	\line(1,0){250}
\end{center}

\clearpage
\section{Co-Processor Instruction Usage}
\paragraph{}This section is similar to the last, but pertains to instructions specific to various defined co-processors. At this time,
no co-processors have been fully developed, nor verified, so this is left here as a placeholder for future revisions of the documentation.
\subsection{Math Unit}
\subsection{Vector Operations Unit}
\subsection{Memory Management Unit}
\subsection{Inter-Processor Communications Unit}

\clearpage

\part{Instructions}
\section{Instruction Definitions}
\subsection{Instruction Format Types}
\paragraph{}This section will talk about the different instructions available in the core processor, their encodings, function, and hazards
they cause or registers they affect in the processor. The Co-Processor instructions are purely generic and allow the implementation of
each Co-Processor to determine how their respective instructions will be decoded. Only the Operation Codes will be defined for each 
Co-Processor slot to allow for more customization.

\newpage
\subsubsection{Integer}
\paragraph{}The integer instructions are the heart of this processor's arithmetic abilities and is vital to ensure fast execution. A semi-strict
adherence to RISC philosophy in this architecture is required to exploit any benefits to this ISA in a real implementation.
\paragraph{}The integer instruction coding with descriptions, is shown in the diagrams below.

\begin{center}
		\begin{tabular}{|c|c|c|c|c|c|}
			\multicolumn{6}{c}{Register/Integer Instruction Format}\\ \hline
				\hspace{2pt} opcode (0x01) \hspace{2pt} & \hspace{5pt} rd \hspace{5pt} &  \hspace{4pt} rsa \hspace{4pt} & \hspace{4pt}rsb  \hspace{4pt}& \hspace{10pt}shft  \hspace{10pt} & \hspace{3pt} aluop \hspace{3pt}   \\	\hline
			31 \hfill 26& 25 \hfill 21 &20 \hfill  16& 15 \hfill  11&10 \hfill   4&3 \hfill   0\\ \hline
		
	\end{tabular}
\end{center}	
\paragraph{Overview:}The Register/Integer Instruction Format is for basic ALU operations, without immediates. Registers RSa and RSb are the two operands,which
are stored in register Rd. The 4 bit ALUOP field denotes the settings for the ALU, to reduce complexity of selecting what operation to choose.
The shft bits are only for the shift amount with the shifting instructions, but unused for other instructions. 
\paragraph{}The following instructions use this encoding:\\
\begin{center}
%	\begin{tabular}{|c|c|c|} \hline
	\begin{longtable}{|l|l|l|}\hline
			Instruction & Description 				& ALU Operation (hex) \\ \hline
			add			& Add 						& 0x0 \\ \hline
			sub 		& Subtract					& 0x1 \\ \hline
			addu 		& Add Unsigned 				& 0x2 \\ \hline
			subu		& Subtract Unsigned 		& 0x3 \\ \hline
			not		& Not			& 0x4 \\ \hline
			and			& And						& 0x5 \\ \hline
			or			& Or						& 0x6 \\ \hline
			xor			& Exclusive Or				& 0x7 \\ \hline
			sal			& Arithmetic Shift Left		& 0x8 \\ \hline
			sar			& Arithmetic Shift Right	& 0x9 \\ \hline
			sll			& Logic Shift Left			& 0xa \\ \hline
			slr			& Logic Shift Right			& 0xb \\ \hline
			comp		& Compare					& 0xc \\ \hline
			Reserved	& n/a						& 0xd \\ \hline
			Reserved	& n/a						& 0xe \\ \hline
			Reserved	& n/a						& 0xf \\ \hline

%	\end{tabular} \\
\end{longtable}
\end{center}
\begin{center}
	\line(1,0){250}
\end{center}

\newpage
\subsubsection{Immediate}

\begin{center}
		\begin{tabular}{|c|c|c|c|c|}
			\multicolumn{5}{c}{Immediate Instruction Format}\\ \hline
				\hspace{2pt} opcode \hspace{2pt} & \hspace{5pt} rd \hspace{5pt} &  \hspace{4pt} rsa \hspace{4pt} & \hspace{10pt}Immediate  \hspace{10pt} & \hspace{3pt} aluop \hspace{3pt}   \\	\hline
			31 \hfill 26& 25 \hfill 21 &20 \hfill  16& 15  \hfill   4&3 \hfill   0\\ \hline
		
	\end{tabular}
\end{center}
\paragraph{Overview:}The Immediate Instruction Format is for ALU operations that require an immediate value. The immediate field
is only 12 bits wide, so in the event that a larger value is required, the Load Immediate Format should be used. There is only a single
source register, RSa, with the other source being the immediate value. The result is stored into register Rd.
\paragraph{}The following instructions use this encoding:
\begin{center}
	\begin{tabular}{|l|l|l|} \hline
			Instruction & Description 						& ALU Operation (hex) \\ \hline
			addi		& Add Immediate						& 0x0 \\ \hline
			subi 		& Subtract Immediate				& 0x1 \\ \hline
			addui 		& Add Unsigned Immediate			& 0x2 \\ \hline
			subui		& Subtract Immediate 				& 0x3 \\ \hline
			noti		& Invert Immediate					& 0x4 \\ \hline
			andi		& And Immediate						& 0x5 \\ \hline
			ori			& Or Immediate						& 0x6 \\ \hline
			xori		& Exclusive Or Immediate			& 0x7 \\ \hline
			sali		& Arithmetic Shift Left				& 0x8 \\ \hline
			sari		& Arithmetic Shift Right			& 0x9 \\ \hline
			slli		& Logic Shift Left 					& 0xa \\ \hline
			slri		& Logic Shift Right					& 0xb \\ \hline
			compi		& Compare Immediate					& 0xc \\ \hline

	\end{tabular} \\
\end{center}
\begin{center}
	\line(1,0){250}
\end{center}

\newpage
\subsubsection{Load/Store}
\begin{center}
		\begin{tabular}{|c|c|c|c|c|}
			\multicolumn{5}{c}{Load Instruction Format}\\ \hline
				\hspace{2pt} opcode \hspace{2pt} & \hspace{5pt} rd \hspace{5pt} &  \hspace{4pt} rsa \hspace{4pt} & \hspace{3pt} funct \hspace{3pt} & \hspace{10pt} Immediate \hspace{10pt}   \\	\hline
				31 \hfill 26& 25 \hfill 21 &20 \hfill  16& 15 \hfill 14  & 13 \hfill 0\\ \hline
		
	\end{tabular}
\end{center}
\paragraph{Overview:}The Load Instruction Format is for reading values from memory into a register. A 14 bit immediate is used
for a relative address calculation with RSa as the base address. The value is then stored into register Rd. The funct field is used
to determine the byte size to read from memory. This encoding is described in the table below. \\
\begin{tabular}{l|l}
		Funct & Description \\ \hline
		00	  & 1 word (32 bits) \\
		01	  & half word (16 bits, upper) \\ 
		10	  & 24 bits, upper \\
		11	  & byte, upper		\\
		\end{tabular}
\paragraph{}The following instructions use this encoding: \\
\begin{center}
	\begin{tabular}{|l|l|l|} \hline
			Instruction & Description 						& Funct\\ \hline
			lw			& Load Word							& 0x0 \\ \hline
			lh			& Load Half Word					& 0x1 \\ \hline
			lb			& Load Byte							& 0x3 \\ \hline
			lth			& Load Three (Bytes, 24 bits)		& 0x2 \\ \hline

\end{tabular}
\end{center}
\begin{center}
	\line(1,0){250}
\end{center}

\begin{center}
		\begin{tabular}{|c|c|c|c|}
				\multicolumn{4}{c}{Load Immediate Instruction Format}\\ \hline
				\hspace{2pt} opcode \hspace{2pt} & \hspace{5pt} rd \hspace{5pt} &  \hspace{4pt} DSEL \hspace{4pt} & \hspace{10pt}Immediate  \hspace{10pt}   \\	\hline
			31 \hfill 26& 25 \hfill 21 &20 \hfill  16& 15  \hfill   0\\ \hline
		
	\end{tabular}
\end{center}

\paragraph{Overview:}The Load Immediate Instruction Format is for loading immediate values into a register.
The DSEL value determines the destination and function of the instruction. This is to allow the selection
of various register files to be updated. The DSEL bit values and their function are shown below. 
The MSB of DSEL determines if the value is unsigned (1) or signed (0). \\
\begin{tabular}{l|l}
		DSEL (bin) & Destination \\ \hline
		0000 & General purpose register file \\
		0001 & System Register File \\
		0010 & Global Co-Processor Register File \\ 
		0011 & General purpose register file (upper 16 bits) \\
		0100 & System Register File (upper 16 bits)\\
		0101 & Global Co-Processor Register File (upper 16 bits)\\ 
		1000 & General purpose register file  (Unsigned)\\
		1001 & System Register File (Unsigned)\\
		1010 & Global Co-Processor Register File (Unsigned)\\ 
		1011 & General purpose register file  (Unsigned, upper 16 bits)\\
		1100 & System Register File (Unsigned, upper 16 bits)\\
		1101 & Global Co-Processor Register File (Unsigned, upper 16 bits)\\ 
		
		\end{tabular}

\paragraph{}The following instructions use this encoding: \\

	\begin{center}
		\begin{tabular}{|l|l|l|} \hline
			Instruction & Description 								& DSEL\\ \hline
			li			& load immediate (gprf)						& 0x0 \\ \hline
			lsi			& load immediate system						& 0x1 \\ \hline
			lgi			& load immediate global						& 0x2 \\ \hline
			lui			& load immediate (gprf)	(upper)				& 0x3 \\ \hline
			lusi		& load immediate system	(upper)				& 0x4 \\ \hline
			lugi		& load immediate global	(upper)				& 0x5 \\ \hline
			lni			& load immediate (gprf)	(unsigned)			& 0x8 \\ \hline
			lnsi		& load immediate system	(unsigned)			& 0x9 \\ \hline
			lngi		& load immediate global	(unsigned)			& 0xa \\ \hline 
			luni		& load immediate (gprf) (unsigned,upper)	& 0xb \\ \hline
			lunsi		& load immediate system	(unsigned,upper		& 0xc \\ \hline
			lungi		& load immediate global	(unsigned,upper)	& 0xd \\ \hline

		\end{tabular}
	\end{center}

\begin{center}
	\line(1,0){250}
\end{center}


\begin{center}
		\begin{tabular}{|c|c|c|c|c|c|}
			\multicolumn{6}{c}{Store Instruction Format}\\ \hline
				\hspace{2pt} opcode \hspace{2pt} & \hspace{5pt} Funct \hspace{5pt} &  \hspace{4pt} Immediate[13:11] \hspace{4pt} & \hspace{4pt}rsa  \hspace{4pt} & \hspace{4pt}rsb  \hspace{4pt} &\hspace{10pt}Immediate[10:0]  \hspace{10pt}     \\	\hline
				31 \hfill 26& 25 \hfill 24 &23 \hfill 21 &20 \hfill  16& 15 \hfill  11&10 \hfill  0\\ \hline
		
	\end{tabular}
\end{center}

\paragraph{Overview:}The Store Instruction Format is for writing register values to memory. A 14 bit immediate is used
for a relative address calculation with RSa as the base address. The value in RSb is then written to memory. The funct field is used
to determine the byte size to write to memory. This encoding is described in the table below. \\
	\begin{tabular}{l|l}
		Funct & Description \\ \hline
		00	  & 1 word (32 bits) \\
		01	  & half word (16 bits, upper) \\ 
		10	  & 24 bits, upper \\
		11	  & byte, upper		\\
	\end{tabular}
\paragraph{}The following instructions use this encoding: \\

	\begin{center}
		\begin{tabular}{|l|l|l|} \hline
			Instruction & Description 						& Funct\\ \hline
			sw			& Store Word							& 0x0 \\ \hline
			sh			& Store Half Word					& 0x1 \\ \hline
			sb			& Store Byte							& 0x3 \\ \hline
			sth			& Store Three (Bytes, 24 bits)		& 0x2 \\ \hline

		\end{tabular}
	\end{center}

\begin{center}
	\line(1,0){250}
\end{center}

\newpage
\subsubsection{Branch/Jump}

\begin{center}
		\begin{tabular}{|c|c|c|c|}
				\multicolumn{4}{c}{Jump Instruction Format}\\ \hline
				\hspace{2pt} opcode \hspace{2pt}  & \hspace{3pt} Immediate[20:16] \hspace{3pt} &  \hspace{4pt} rsa \hspace{4pt} & \hspace{10pt} Immediate[15:0]  \hspace{10pt}   \\	\hline
							31 \hfill 26& 25  \hfill 21 &20 \hfill  16& 15 \hfill   0\\ \hline
		
	\end{tabular}
\end{center}	
\paragraph{Overview:} The Jump Instruction format is for changing the Program Counter. 
All Jumps are relative, except Jump Register Instructions,
due to the non byte aligned size (20 bits) of the immediate field. Jumps use the value in register RSa to calculate
the new PC. For pure relative jumps, the Zero register is used as the base. For jump instructions that link the program counter, 
the PC is stored in RA0, or R4 in the General Purpose Register file.
\paragraph{}The following instructions use this encoding: \\
\begin{center}
	\begin{tabular}{|l|l|} \hline
			Instruction & Description 				\\ \hline
			j			& Jump						\\ \hline
			jal			& Jump and Link				\\ \hline
			jr			& Jump Register				\\ \hline
			jrl			& Jump Register and Link	\\ \hline


\end{tabular}
\end{center}
\begin{center}
	\line(1,0){250}
\end{center}


\begin{center}
		\begin{tabular}{|c|c|c|c|c|c|}
			\multicolumn{6}{c}{Branch Instruction Format}\\ \hline
				\hspace{2pt} opcode \hspace{2pt} & \hspace{5pt} Immediate[13:9] \hspace{5pt} &  \hspace{4pt} rsa \hspace{4pt} & \hspace{4pt}rsb  \hspace{4pt}& \hspace{10pt}Immediate[8:0]  \hspace{10pt} & \hspace{3pt} funct \hspace{3pt}   \\	\hline
			31 \hfill 26& 25 \hfill 21 &20 \hfill  16& 15 \hfill  11&10 \hfill   2&1 \hfill   0\\ \hline
		
	\end{tabular}
\end{center}
\paragraph{Overview:} The Branch Instruction Format is used for conditionally changing the Program Counter.
The PC is only updated with whichever operation is indicated as being true or false.\\
\paragraph{}The following instructions use this encoding: \\
\begin{center}
	\begin{tabular}{|l|l|l|} \hline
			Instruction & Description 				& Funct\\ \hline
			beq			& Branch if Equal			& 00 \\ \hline
			bne			& Branch if not Equal		& 01 \\ \hline
			bgt			& Branch if Greater than	& 10 \\ \hline
			blt			& Branch if Less than		& 11 \\ \hline

\end{tabular}
\end{center}
\begin{center}
	\line(1,0){250}
\end{center}

\newpage
\subsubsection{System Instructions}
\begin{center}
	\begin{tabular}{|c|c|c|c|c|}
		\multicolumn{5}{c}{System Instruction Format}\\ \hline
			\hspace{2pt} opcode \hspace{2pt} & \hspace{5pt} rd \hspace{5pt} &  \hspace{4pt} rsa \hspace{4pt} & \hspace{10pt}Funct  \hspace{10pt} & \hspace{8pt} Immediate[7:0] \hspace{8pt}   \\	\hline
		31 \hfill 26& 25 \hfill 21 &20 \hfill  16& 15 \hfill  8&7 \hfill   0\\ \hline
		
	\end{tabular}
\end{center}
\paragraph{Overview:} The System Instruction Format is used for various system controls. Their functions
vary, and are described in more detail in their individual listings.
\paragraph{}The following instructions use this encoding: \\
\begin{center}
	\begin{tabular}{|l|l|l|l|} \hline
			Instruction & Description 						& Funct & Uses Immediate\\ \hline
			syscall		& System Call						& 0x00  & yes \\ \hline
			sysret		& System Return						& 0x01  & no  \\ \hline
			stspr		& Store Special Purpose Register	& 0x02  & yes \\ \hline
			ldspr		& Load Special Purpose Register		& 0x03  & yes \\ \hline
			sync		& Synchronize memory/Flush Pipeline  & 0x04  & no  \\ \hline
			lock		& Lock Memory						& 0x05  & yes \\ \hline
			test		& Test Lock 						& 0x06  & yes \\ \hline
			pmir		& Permission Increase Request		& 0x07  & yes \\ \hline
			pmd			& Permission Decrease				& 0x08  & yes \\ \hline


\end{tabular}
\end{center}
\begin{center}
	\line(1,0){250}
\end{center}
\newpage
\subsubsection{Co-Processor}

\paragraph{}Since the co-processors can have any implementation, only the opcode is required. However, adhering to similar 
formatting of the instruction formats provided in previous sections is imperative and the Fusion-Core foundation 
will not provide an accepted co-processor ID number. More information about co-processors can be accessed in the Co-Processor section.
\begin{center}
	\line(1,0){250}
\end{center}

\newpage
\subsection{Opcodes}
\paragraph{} The opcodes are designed to provide a simple decode unit for the processor. The figure below shows the breakdown
within the opcode.
	\begin{center}
		\begin{tabular}{|c|c|c|}
			\hline
				CPEN & Mem/PC[1:0] & Reg Operands[2:0]	\\ \hline
				5 & 4 \hfill 3 & 2 \hfill 0 \\ \hline

		\end{tabular}
	\end{center}
\paragraph{} The most significant bit of the opcode field determines if the instruction is a Co-processor instruction or not.
This is to offload decoding of co-processor instructions for the individual co-processors, and not cause any requirements
for future developments of co-processor instruction decoding. 
\paragraph{} Bits 3 and 4 of the opcode field are for determining memory access, changing the program counter, and linking
the return address to register RA. A possible way to generate signals for decoding the opcode is provided below in verilog. \\

\texttt{pc\_change = ~opcode[4]}\\
\texttt{pc\_link = opcode[4] ~| opcode[3]}\\
\texttt{mem\_access = opcode[4] \& opcode[3]}\\

\paragraph{} The lower 3 bits of the opcode field determine the operand usage of an instruction, including Rd, RSa, RSb, and immediates.
A possible way to generate signals for decoding the opcode is provided below in verilog. \\

\texttt{use\_rd = ( opcode[1] | opcode[0] ) \& ( ~opcode[2] | opcode[1] ) }\\
\texttt{use\_rsa = opcode[2] | opcode[1] }\\
\texttt{use\_rsb = ( opcode[2] \& opcode[0] ) | ( opcode[1] \& opcode[0] ) }\\
\texttt{use\_immediate = opcode[2] | ~opcode[1] }\\



\subsubsection{List of OPCodes}
	\begin{center}
		\begin{tabular}{|l|l|}
			\hline
			Op Code & Description 				 	\\ \hline
			010011	& Register ALU instructions	 	\\ \hline
			010000	& Load Immediate instructions 	\\ \hline
			010110	& Immediate ALU instructions	\\ \hline
			001101	& Branch Instructions			\\ \hline
			001100	& Jump Instructions				\\ \hline
			000100	& Jump and Link Instructions	\\ \hline
			100000	& System Instructions			\\ \hline
			011110	& Load Instructions				\\ \hline
			011101	& Store Instructions			\\ \hline
			11XXXX	& Co-Processor Instructions		\\ \hline

		\end{tabular}
	\end{center}
\paragraph{} Each bit of the opcode directly affects the resources required for each instruction. This choice was made 
in hopes of reducing decode complexity. The MSB of the opcode selects if the instruction is for the co-processor,
or main core. This makes it incredibly easy to determine which core the instruction goes to, without sacrificing
more opcode bits. The LSB of the opcode selects ALU usage, the second to last bit selects the immediate field.
\paragraph{} However, in some cases the more significant bit will modify the functions of the later bits. The 3rd to last
bit in the opcode selects a change in PC, but the following bits do not correspond to the previous meanings mentioned. The
4th bit denotes a system instruction, and the 5th bit denotes memory access. 
\newpage


\subsection{List of Instructions}

	\begin{center}
		\begin{longtable}{|l|l|c|}
			\multicolumn{3}{c}{Instruction Instruction Summary Table} \\ \hline
			\multicolumn{3}{|c|}{Integer Instructions} \\ \hline
			Instruction & Function 										& Binary \\ \hline
			add			& Rd = RSa + RSb 								& 000001dddddaaaaabbbbbxxxxxxx0000 \\
			sub			& Rd = RSa - RSb 								& 000001dddddaaaaabbbbbxxxxxxx0001 \\
			addu		& Rd = RSa + RSb (Unsigned) 					& 000001dddddaaaaabbbbbxxxxxxx0010 \\
			subu		& Rd = RSa - RSb (Unsigned)					 	& 000001dddddaaaaabbbbbxxxxxxx0011 \\
			not		& Rd = !RSa	 									& 000001dddddaaaaaxxxxxxxxxxxx0100 \\
			and			& Rd = RSa \& RSb								& 000001dddddaaaaabbbbbxxxxxxx0101 \\
			or			& Rd = RSa $|$ RSb 								& 000001dddddaaaaabbbbbxxxxxxx0110 \\
			xor			& Rd = RSa $\oplus$ RSb 						& 000001dddddaaaaabbbbbxxxxxxx0111 \\
			sal			& Rd = RSa $\ll$ RSb 							& 000001dddddaaaaabbbbbsssssss1000 \\
			sar			& Rd = RSa $\gg$ RSb 							& 000001dddddaaaaabbbbbsssssss1001 \\
			sll			& Rd = RSa $\ll$ RSb 							& 000001dddddaaaaabbbbbsssssss1010 \\
			slr			& Rd = RSa $ \ggg  $ RSb 						& 000001dddddaaaaabbbbbsssssss1011 \\
			comp		& Rd = (RSa == RSb);(RSa $>$ RSb);(RSa $<$ RSb)	& 000001dddddaaaaaxxxxxsssssss1100 \\
			\hline
			\multicolumn{3}{|c|}{Immediate Instructions} \\ \hline
			addi		& Rd = RSa + Imm 								& 000011dddddaaaaaiiiiiiiiiiii0000 \\
			subi		& Rd = RSa - Imm 								& 000011dddddaaaaaiiiiiiiiiiii0001 \\
			addui		& Rd = RSa + Imm (Unsigned)						& 000011dddddaaaaaiiiiiiiiiiii0010 \\
			subui		& Rd = RSa - Imm (Unsigned)					 	& 000011dddddaaaaaiiiiiiiiiiii0011 \\
			noti		& Rd = ! Imm									& 000011ddddd00000iiiiiiiiiiii0100 \\
			andi		& Rd = RSa \& Imm								& 000011dddddaaaaaiiiiiiiiiiii0101 \\
			ori			& Rd = RSa $|$ Imm 								& 000011dddddaaaaaiiiiiiiiiiii0110 \\
			xori		& Rd = RSa $\oplus$ Imm 						& 000011dddddaaaaaiiiiiiiiiiii0111 \\
			sali		& Rd = RSa $\ll$ Imm 							& 000011dddddaaaaaiiiiiiiiiiii1000 \\
			sari		& Rd = RSa $\gg$ Imm 							& 000011dddddaaaaaiiiiiiiiiiii1001 \\
			slli		& Rd = RSa $\ll$ Imm 							& 000011dddddaaaaaiiiiiiiiiiii1010 \\
			slri		& Rd = RSa $ \ggg  $ Imm 						& 000011dddddaaaaaiiiiiiiiiiii1011 \\
			compi		& Rd = (RSa == Imm);(RSa $>$ Imm);(RSa $<$ Imm)	& 000011dddddaaaaaiiiiiiiiiiii1100 \\
			\hline
			\multicolumn{3}{|c|}{Load Instructions} \\ \hline
			lw			& Rd $<$- Imm(RSa)								& 010010dddddaaaaa00iiiiiiiiiiiiii \\
			lh			& Rd $<$- Imm(RSa)								& 010010dddddaaaaa01iiiiiiiiiiiiii \\
			lb			& Rd $<$- Imm(RSa)								& 010010dddddaaaaa11iiiiiiiiiiiiii \\
			lth			& Rd $<$- Imm(RSa)								& 010010dddddaaaaa10iiiiiiiiiiiiii \\
			\hline
			\multicolumn{3}{|c|}{Load Immediate Instructions} \\ \hline
			li			& (GPREGF) Rd = Imm								& 000010ddddd0000iiiiiiiiiiiiiiiii \\
			lsi			& (SYSREGF) Rd = Imm							& 000010ddddd0001iiiiiiiiiiiiiiiii \\
			lgi			& (GLBREGF) Rd = Imm								& 000010ddddd0010iiiiiiiiiiiiiiiii \\
			\hline \newpage
			\hline
			\multicolumn{3}{|c|}{Load Immediate Instructions (cont.)} \\ \hline
			Instruction & Function 										& Binary \\ \hline
			lui			& (GPREGF) Rd = Imm	(upper 16 bits)				& 000010ddddd0011iiiiiiiiiiiiiiiii \\
			lusi		& (SYSREGF) Rd = Imm (upper 16 bits)			& 000010ddddd0100iiiiiiiiiiiiiiiii \\
			lugi		& (GLBREGF) Rd = Imm	 (upper 16 bits)			& 000010ddddd0101iiiiiiiiiiiiiiiii \\
			lni			& (GPREGF) Rd = Imm	(unsigned)					& 000010ddddd1000iiiiiiiiiiiiiiiii \\
			lnsi		& (SYSREGF) Rd = Imm (unsigned)					& 000010ddddd1001iiiiiiiiiiiiiiiii \\
			lngi		& (GLBREGF) Rd = Imm (unsigned)					& 000010ddddd1010iiiiiiiiiiiiiiiii \\
			luni		& (GPREGF) Rd = Imm	 (upper 16, unsigned)		& 000010ddddd1011iiiiiiiiiiiiiiiii \\
			lunsi		& (SYSREGF) Rd = Imm (upper 16, unsigned)		& 000010ddddd1100iiiiiiiiiiiiiiiii \\
			lungi		& (GLBREGF) Rd = Imm	(upper 16, unsigned)		& 000010ddddd1101iiiiiiiiiiiiiiiii \\
			\hline
			\multicolumn{3}{|c|}{Store Instructions} \\ \hline

			sw			& RSb -$>$ Imm(RSa)								& 01101000iiiaaaaabbbbbiiiiiiiiiii\\
			sh			& RSb -$>$ Imm(RSa)								& 01101001iiiaaaaabbbbbiiiiiiiiiii\\
			sb			& RSb -$>$ Imm(RSa)								& 01101011iiiaaaaabbbbbiiiiiiiiiii\\
			sth			& RSb -$>$ Imm(RSa)								& 01101010iiiaaaaabbbbbiiiiiiiiiii\\
			\hline
			\multicolumn{3}{|c|}{Jump Instructions} \\ \hline		
			j			& Next PC $<$- (R0 + Imm)						& 000110iiiiiaaaaaiiiiiiiiiiiiiiii\\
			jal			& Next PC $<$- (R0 + Imm); RA0 <- PC 			& 000111iiiiiaaaaaiiiiiiiiiiiiiiii\\
			jr			& Next PC $<$- (RSa + Imm)						& 000110iiiiiaaaaaiiiiiiiiiiiiiiii\\
			jrl			& Next PC $<$- (RSa + Imm); RA0 <- PC			& 000111iiiiiaaaaaiiiiiiiiiiiiiiii\\
			\hline
			\multicolumn{3}{|c|}{Branch Instructions} \\ \hline
			beq			& Next PC $<$- (RSa == RSb) ? PC+Imm : PC+4		& 000101iiiiiaaaaabbbbbiiiiiiiii00\\
			bne			& Next PC $<$- (RSa != RSb) ? PC+Imm : PC+4		& 000101iiiiiaaaaabbbbbiiiiiiiii01\\
			bgt			& Next PC $<$- (RSa $>$ RSb) ? PC+Imm : PC+4	& 000101iiiiiaaaaabbbbbiiiiiiiii10\\
			blt			& Next PC $<$- (RSa $<$ RSb) ? PC+Imm : PC+4	& 000101iiiiiaaaaabbbbbiiiiiiiii11\\
			\hline
			\multicolumn{3}{|c|}{System Instructions} \\ \hline
			syscall		& System Call (Raise Privilege)					& 00100dddddaaaaaa00000000iiiiiiii\\
			sysret		& System Return (Lower Privilege)				& 00100dddddaaaaaa00000001iiiiiiii\\
			stspr		& (SYSREGF) Rd $<$- RSa							& 00100dddddaaaaaa00000010xxxxxxxx\\
			ldspr		& (SYSREGF) RSa -$>$ Rd							& 00100dddddaaaaaa00000011xxxxxxxx\\
			sync		& Flush Pipeline								& 00100xxxxxxxxxxx00000100xxxxxxxx\\
			pmir		& PML + Imm ?									& 00100xxxxxxxxxxx00000111iiiiiiii\\
			pmd			& PML - Imm ?									& 00100xxxxxxxxxxx00001000iiiiiiii\\
			\hline
		\end{longtable}
	\end{center}


\subsubsection{Co-Processor}
\paragraph{} More information on co-processor instructions will be available
in the future. This section is still under development.

\clearpage
\section{Exceptions and Interrupts}
\paragraph{} This ISA differentiates between Exceptions and Interrupts based on function. Interrupts are vectored, to allow
the running process to define what should be done. Exceptions are more based in hardware, and require minimal setup in software.

\subsection{Exceptions}
\paragraph{} This section is still under development. 
\subsubsection{User Level}
\subsubsection{Supervisor Level}

\begin{center}
	\line(1,0){250}
\end{center}

\subsection{Interrupts}
\paragraph{} This section is still under development.
\subsubsection{User Level}
\subsubsection{Supervisor Level}

\begin{center}
	\line(1,0){250}
\end{center}

\clearpage
\part{Co-Processors}
\section{Co-Processor Overview}
\paragraph{}Co-processors are the main point of the Fusion-Core architecture. As the ISA strictly specifies the main core for code compatibility,
the implementation is free to use whichever co-processors that would be necessary for an application. As speeding up the common case is the main goal in CPU design,
these co-processors should enhance the common case for the specific implementation. The ISA has no specific restrictions for a co-processor, but for non-memory
mapped co-processors, they should be created as dictated by the interface specified below.

\paragraph{} Examples of co-processors could be hardware acceleration for vector instructions, encryption, floating point, or communication.
The co-processor can be as complex, or as simple as one requires. There is no limitation for what kind of co-processor that could be used, 
only the number of co-processors that can utilize the opcode space allocated for co-processors. The co-processors can also share the same opcode, which
can help create vectorized instruction units such as GPU like co-processors, coupled directly to the main core's instruction flow. The programmer would
be able to exploit more locality in programming as switching to different sections of memory is not required. It should be noted that this last sentence is
not a requirement, but just a possibility and the current implementation of the default case for the GNU Binutils port. This can easily be modified with a 
custom linker script.

\paragraph{} The reasoning behind co-processors use is to allow for code compatibility between nearly all implementations of the Fusion-Core ISA, and
allowing for an enhancement to the common case of a processor's use. Most architectures do not allow for radical implementations in the architecture itself,
only the micro-architecture. And by keeping complex instructions away from the main core, a simple decode unit can be created to allow for potentially faster
pipelining. 

\subsection{ Co-Processor Interface}

\paragraph{ Co-Processor Opcode space} Co-processor instructions are enabled through use of the most significant bit of the opcode field. 
The processor should be able to distinguish between main core and co-processor instructions quickly, and a single bit is the simplest way of doing so.
The other 5 bits are completely usable for whichever purpose the implementation could desire.

\paragraph{} In order to let the programmer know what co-processors are available, the opcode registration table should contain the co-processor ID (CPID).
This ID is provided by the Fusion-Core developers. Please send an email to cpid@fusion-core.org for making requests for a new co-processor.
This ID will be implemented to allow for proper disassembly and assembly of the instructions in the GNU Binutils port. It is also possible to not use an official
CPID, however no support will be given to the developers of the co-processor.

\subsubsection{ Decode unit Connections}
\subsubsection{ Co-Processor Conventions}
\subsubsection{Register Connections}
\subsection{Interface Connection Definitions}
\subsection{Adding custom Co-Processor}
\subsection{List of Co-Processors}
\subsubsection{Floating Point}
\subsubsection{System Unit}
\subsubsection{Memory Management Unit}
\subsubsection{Multiprocessor Communication Unit}

\section{Global Register File}
\paragraph{}The Global Register File is for simple message passing, and creating locks between co-processors.
As it may be necessary to wait for a value to be computed by a co-processor, or lock specific parts of memory,
the Global Register File creates an interface for ease of use between processing units.
\paragraph{}This section is under development and will be updated to explain the connections and registers available.

\section{Recommended Co-Processors}
\paragraph{About} This section will cover some basic co-processors that have been approved and assigned co-processor IDs. The full list of approved co-processors will be included in a separate document. 
\subsection{Math Unit}
\subsubsection{Registers}
\subsubsection{Instructions}
\subsection{System Unit}
\subsubsection{Registers}
\subsubsection{Instructions}
\subsection{Memory Management Unit}
\subsubsection{Registers}
\subsubsection{Instructions}
\subsection{Inter-Processor Communications Unit}
\subsubsection{Registers}
\subsubsection{Instructions}






\end{document}
