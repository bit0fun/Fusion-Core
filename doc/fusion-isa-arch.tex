\documentclass[letterpaper]{article}
\usepackage{graphicx}
\usepackage[left=1.00in, right=1.00in, top=1.00in, bottom=1.00in]{geometry}
\usepackage{hyperref}
\hypersetup{colorlinks=false,linktoc=all}



\title{Fusion-Core ISA Definition: Revision 0.1}
\author{Dylan Wadler}

\begin{document}
\maketitle
\newpage
\tableofcontents


\newpage
\section{Changelog}
\paragraph{Version 0.1} Initial Definition of the Instruction Set Architecture

\section{Introduction}

\subsection{About}
\paragraph{}The Fusion-Core ISA is dedicated to creating an easily expandible architecture without altering
the instruction set. By use of defining an easy interface with a simple core instruction set, this allows for
more freedom in implementation. High end processors and microcontrollers would only have slight varaitions in
configuration, as their core would remain identical save for easy to maintain and scalable co-processors.
\paragraph{}At this moment in time, the Fusion-Core ISA is only a 32 bit ISA. Due to the focus on co-processors,
older implementations could easily be modified to include 64 bit operations.


\subsection{Goals}
\subsection{Conventions}



\section{Register File Defintions}
\subsection{General Purpose Registers}
\subsubsection{Definition}
\subsubsection{Usage}
\subsection{Speial Registers}
\subsubsection{Control Registers}
\subsubsection{Supervisor Registers}


\section{Instruction Defintions}
\subsection{Instruction Types}
\subsubsection{Integer}
\subsubsection{Immediate}
\subsubsection{Load\/Store}
\subsubsection{Branch\/Jump}
\subsubsection{Floating Point}
\subsubsection{Atomic}
\subsubsection{System}
\subsubsection{Co-Processor}
\subsubsection{Custom}
\subsection{List of Instructons}
\subsubsection{Integer}
\subsubsection{Immediate}
\subsubsection{Load\/Store}
\subsubsection{Branch\/Jump}
\subsubsection{Floating Point}
\subsubsection{Atomic}
\subsubsection{System}
\subsubsection{Co-Processor}
\subsubsection{Custom}


\section{Exceptions and Interrupts}
\subsection{Exceptions}


\subsection{Interrupts}
\subsubsection{User Level}
\subsubsection{Supervisor Level}

\section{Co-Processor Interface}
\subsection{Interface Connection Definitions}
\subsection{Adding custom Co-Processor}
\subsection{List of Co-Processors}

\section{Memory Map}
\textbf{NOTE}
This section may not remain and be left up to the implementation. Do not use without consulting
the documentation of the implementation.
\section{Programming Conventions}
\subsection{Register Usage}
\subsection{Memory Locations for Vector Table}
\subsubsection{Interrupt Vector Table}
\subsubsection{Exception Vector Table}


\end{document}
